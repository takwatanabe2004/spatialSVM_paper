%!TEX root = paper.tex
%*****************************************************************************%
% 						Master initialization file
%*****************************************************************************%
%===================================================================%
% configurations contained in the template files from elsevier
%===================================================================%
\usepackage{amssymb}
\biboptions{authoryear,sort}

%*****************************************************************************%
%		 			Packages used for the report
%*****************************************************************************%
\usepackage{amsmath}
\usepackage{xspace}
\usepackage{color}
\usepackage{xcolor}
\usepackage[font=footnotesize]{caption} %<- (01/20/2014) neuroimage's online compiler uses normal fontsize...force it to the default on my pc
%\usepackage{caption} %<- (10/31/2013) this needed to be loaded for EES tex compiler to run!
%\usepackage[font=footnotesize]{subcaption}
\usepackage{subcaption}
\usepackage{url}
\usepackage{comment}
\usepackage{verbatim}
\usepackage{amsthm} 
\usepackage{framed} 
\usepackage{epsfig}
\usepackage{graphicx}
\usepackage{bm}
\usepackage{bbm} 

\usepackage{rotating}
\usepackage{ulem}
\usepackage{algorithm}
\usepackage{algpseudocode}
\usepackage{soul} 
\usepackage{setspace} 
\usepackage{soul} 
\usepackage{afterpage} 
\usepackage{mathabx} 

%====================================%
% Bookmark stuffs
%====================================%
\usepackage[open,openlevel=2,numbered]{bookmark}
\hypersetup{pdftex,  
  breaklinks=true, 
  colorlinks=true,
  linkcolor=dark_blue,
  citecolor=deep_blue,
  pdfpagemode=UseNone, % <- (01/10/2014) to suppress bookmark when opening in adobe http://tex.stackexchange.com/questions/50814/how-to-disable-initial-view-of-pdf-bookmarks-panel
}

%====================================%
% tikz packages
%====================================%
\usepackage{etex}
\usepackage{pgf}
\usepackage{tikz}
\usetikzlibrary{automata}
\pgfdeclarelayer{background}
\pgfsetlayers{main}
\usetikzlibrary{shapes}
\usetikzlibrary{matrix} 

%*****************************************************************************%
%		 			 my custom math commands
%*****************************************************************************%
\newcommand{\inmath}	{\ensuremath}
\newcommand{\xmath}[1]	{\ensuremath{#1}\xspace}
\newcommand{\bmath}[1]	{\xmath{\bm{#1}}}	% needs \usepackage{bm}

\newcommand{\argmax}[1] 	{\operatorname*{arg\,max}_{#1}}
\newcommand{\argmin}[1] 	{\operatorname*{arg\,min}_{#1}}

\newcommand{\minimize}[1] 	{\operatorname*{minimize}_{#1}}

\newcommand{\reals}			{\xmath{\mathbb{R}}}
\newcommand{\naturals}		{\xmath{\mathbb{N}}}

\newcommand{\iid}		{\stackrel{iid}{\sim}}
\newcommand{\icomplex}	{\xmath{\imath}}

\newcommand{\brackp}[1]		{\xmath{\left(#1\right)}}
\newcommand{\brackb}[1]		{\xmath{\left[#1\right]}}
\newcommand{\brackB}[1]		{\xmath{\left\{#1\right\}}}
\newcommand{\brackv}[1]		{\xmath{\left|#1\right|}}
\newcommand{\brackV}[1]		{\xmath{\left\|#1\right\|}}

% my favorite style of matrix
\newcommand{\bmat}		{\begin{bmatrix}} % for matrix
\newcommand{\emat}		{\end{bmatrix}}

% some "brackets" and "braces"
\newcommand{\braces}[1]	{\xmath{\left\{#1\right\}}}
\newcommand{\bracesb}[1]	{\xmath{\Big\{#1\Big\}}}
\newcommand{\bracesbb}[1]	{\xmath{\Bigg\{#1\Bigg\}}}

\newcommand{\abs}[1]	{\xmath{\left| #1 \right|}}
\newcommand{\norm}[1]	{\xmath{\left\| #1 \right\|}}
\newcommand{\normb}[1]	{\xmath{\Big\| #1 \Big\|}}
\newcommand{\normbb}[1]	{\xmath{\Bigg\| #1 \Bigg\|}}
\newcommand{\floor}[1]	{\xmath{\left\lfloor #1 \right\rfloor}}
\newcommand{\ceil}[1]	{\xmath{\left\lceil #1 \right\rceil}}
\newcommand{\inprod}[1]	{\xmath{\mathop{\langle #1\rangle}\nolimits}}
\newcommand{\Inprod}[1]	{\xmath{\left\langle #1\right\rangle}}

\newcommand{\musec}		{\xmath{\mu\mathrm{sec}}}

\newcommand{\eexp}		{\xmath{\mathinner{\mathrm{e}}}}
\newcommand{\Expi}[1]	{\xmath{\eexp^{\icomplex #1}}}
\newcommand{\expi}[1]	{\xmath{\exp\of{\icomplex #1}}}
\newcommand{\Expni}[1]	{\xmath{\eexp^{-\icomplex #1}}}
\newcommand{\expni}[1]	{\xmath{\exp\of{-\icomplex #1}}}
\newcommand{\Expo}[1]	{\xmath{\eexp^{#1}}}
\newcommand{\expo}[1]	{\xmath{\exp\of{#1}}}
\newcommand{\Expn}[1]	{\xmath{\eexp^{-#1}}}
\newcommand{\expn}[1]	{\xmath{\exp\of{-#1}}}
\newcommand{\Exp}		{\expo}	
\newcommand{\Cos}[1]	{\xmath{\cos\of{#1}}}
\newcommand{\Sin}[1]	{\xmath{\sin\of{#1}}}

\newcommand{\undertext}[2]	{\underbrace{#1}_{\mbox{#2}}}
\newcommand{\dsqrt}[1]	{\xmath{\sqrt{\displaystyle #1}}}
\newcommand{\xfrac}[2]	{\xmath{\frac{#1}{#2}}}
\newcommand{\Half}[1][1]	{\xfrac{#1}{2}}
\newcommand{\half}[1][1]	{\xmath{{\scriptstyle \frac{#1}{2}}}}
\newcommand{\Frac}[2]	{\xmath{{#1}/{#2}}}
\newcommand{\pfrac}[3][]	{\xmath{\!\paren{#1\frac{#2}{#3}}}}
\newcommand{\pFrac}[3][]	{\xmath{\!\paren{#1\Frac{#2}{#3}}}}
\newcommand{\dpfrac}[3][]	{\xmath{\!\paren{#1\dfrac{#2}{#3}}}}
\def\dfrac#1#2{\xmath{\displaystyle\frac{#1}{#2}}}
\def\implies	{\xmath{\Longrightarrow}} 

\newcommand{\eref}[1]	{(\protect\ref{#1})}
\newcommand{\fref}[2][]	{Fig.~\protect\ref{#2}#1}
\newcommand{\conj}		{\ensuremath{^*}}
\newcommand{\conjb}[1]	{\xmath{\brak{#1}\conj}} 
\newcommand{\degr}		{\xmath{^{\circ}}}

\newcommand{\conv}		{\ensuremath{\ast}}
\newcommand{\cconv}		{\ensuremath{\ast\mbox{$\!\!$}\ast}}
\newcommand{\ccconv}	{\ensuremath{\mathbin{\ast \mbox{$\!\!$} \ast \mbox{$\!\!$}\ast}}}

\newcommand{\inv}		{\ensuremath{^{-\!1}}}
\newcommand{\powten}[1]	{\ensuremath{\cdot10^{#1}}}

\newcommand{\dstyle}{\displaystyle}

% integrals and derivatives
\newcommand{\negint}	{-\!\!\int\!}
\newcommand{\intinf}	{\int_{\!-\infty}^{\infty}}
\newcommand{\iintinf}	{\intinf\intinf}
\newcommand{\der}[1]	{\xmath{\mathop{\mathrm{d}#1}\nolimits}}
\newcommand{\dxy}		{\der{x}\der{y}}
\newcommand{\dxyz}		{\der{x}\der{y}\der{z}}

\newcommand{\ddt}[1][t]	{\xfrac{\der{}}{\der{#1}}}
\newcommand{\dder}[2]	{\xfrac{\der{}^{#2}}{\der{#1}^{#2}}}
\newcommand{\fpart}[2][]	{\xfrac{\partial #1}{\partial #2}}
\newcommand{\ffpart}[2][]	{\xfrac{\partial^2 #1}{\partial #2^2}}
\newcommand{\fffpart}[2][]	{\xfrac{\partial^3 #1}{\partial #2^3}}
\newcommand{\ffpartt}[2]	{\xfrac{\partial^2}{\partial #1 \; \partial #2}}

%------Not convinced these are superior to \text-------
\newcommand{\mathword}[3][]	{\,\inmath{\mathrm{#2}#1\of{#3}}}
\newcommand{\wordbrace}[3][]{\,\inmath{\mathrm{#2}#1\!\braces{#3}}}
\newcommand{\Poisson}[1]	{\wordbrace{Poisson}{#1}}

\newcommand{\Normal}[1]	{{\mathcal N}\of{#1}}
\newcommand{\diag}[1]	{\wordbrace{diag}{#1}}
\newcommand{\var}[1]	{\wordbrace{Var}{#1}}
\newcommand{\cov}[1]	{\wordbrace{Cov}{#1}}
\newcommand{\sinc}		{\Ofun{\mathrm{sinc}}}
\newcommand{\rect}[1] 	{\mathword{rect}{#1}}
\newcommand{\sgn}[1] 	{\mathword{sgn}{#1}}
\newcommand{\real}[1]	{\mathword{real}{#1}}
\newcommand{\imag}[1]	{\mathword{imag}{#1}}
%\-----------------------------------------------%


%====================================%
% mathematical environments
%====================================%
\newcommand{\ba}[1]		{\left[ \begin{array}{#1}}
\newcommand{\ea}		{\end{array} \right]}
\newcommand{\be}		{\begin{equation}}
\newcommand{\ee}[1]		{\label{#1}\end{equation}}

%*****************************************************************************%
%                         set custom commands
%*****************************************************************************%
%*****************************************************************************%
%                         set custom commands
%*****************************************************************************%
%====================================%
% commands contained in the original template files
%====================================%
\newcommand{\sun}{\ensuremath{\odot}} % sun symbol is \sun

%====================================%
% abbreviations
%====================================%
\newcommand{\etal}		{\emph{et al\@.}\xspace}
\newcommand{\ie}		{\emph{i.e\@.}\xspace}
\newcommand{\eg}		{\emph{e.g\@.}\xspace}
\newcommand{\cf}		{\emph{cf\@.}\xspace}
\newcommand{\defacto}	{\emph{de facto}\xspace}
\newcommand{\adhoc}		{\emph{ad hoc}\xspace}
\newcommand{\apriori}	{\emph{a priori}\xspace}

%=============================================================================%
% Custom theorem contained in original template
% - All of the following have the same numbering system as theorem.
% - The definitions below were included from the template file
%=============================================================================%
\theoremstyle{plain}
\newtheorem{theorem}{Theorem}
\newtheorem{prop}[theorem]{Proposition}
\newtheorem{corollary}[theorem]{Corollary}
\newtheorem{lemma}[theorem]{Lemma}
\newtheorem{question}[theorem]{Question}
\newtheorem{conjecture}[theorem]{Conjecture}
\newtheorem{assumption}[theorem]{Assumption}

\theoremstyle{definition}
\newtheorem{definition}[theorem]{Definition}
\newtheorem{notation}[theorem]{Notation}
\newtheorem{condition}[theorem]{Condition}
\newtheorem{example}[theorem]{Example}
\newtheorem{introduction}[theorem]{Introduction}

\theoremstyle{remark}
\newtheorem{remark}[theorem]{Remark}

%====================================%
% for figures
%====================================%
% these will be updated several times in the paper through \renewcommand
\newcommand{\imwidth}  {0.23\linewidth}
\newcommand{\imheight} {0.21\linewidth}
\newcommand{\HSPACE}{\hspace{10pt}}
\newcommand{\VSPACE}{\vspace{10pt}}
\newcommand{\VSPACEE}{\vspace{10pt}}
\newcommand{\imwidthh}  {0.23\linewidth}
\newcommand{\imheightt}  {0.23\linewidth}
\newcommand{\myfbox}[1]{\setlength{\fboxrule}{1.25pt}\fbox{#1}\setlength{\fboxrule}{0.25pt}}

%====================================%
% color commands
%====================================%
\definecolor{DarkBlue}{rgb}{0,0.2,0.65}
\definecolor{black}{rgb}{0,0,0}
\definecolor{myblack}{rgb}{0,0,0}
\definecolor{myred}{rgb}{0.85,0,0}
\definecolor{mygreen}{rgb}{0,0.6,0}
\definecolor{myblue}{rgb}{0,0,0.5}

\definecolor{myorange}{rgb}{1,0.6,0.2}
%\definecolor{myorange}{rgb}{0.9,0.3,0}
\definecolor{myyellow}{rgb}{1,1,0}
\definecolor{mypurple}{rgb}{.7,0,.7}
\definecolor{mywhite}{rgb}{1,1,1}

\newcommand{\dored}[1]		{{\color{myred}{#1}}}
\newcommand{\doredbf}[1]		{{\color{red}{\textbf{#1}}}}
\newcommand{\doblue}[1]		{{\color{blue}{#1}}}
\newcommand{\dobluebf}[1]		{{\color{blue}{\textbf{#1}}}}
\newcommand{\doorange}[1]		{{\color{myorange}{#1}}}
\newcommand{\dogreen}[1]		{{\color{mygreen}{#1}}}
\newcommand{\dogreenbf}[1]		{{\color{mygreen}{\textbf{#1}}}}
\newcommand{\dopurple}[1]		{{\color{mypurple}{#1}}}
\newcommand{\dowhite}[1]		{{\color{white}{#1}}}

% http://tex.stackexchange.com/questions/9363/how-does-one-insert-a-backslash-or-a-tilde-into-latex
\newcommand{\mytilde}{\raise.17ex\hbox{$\scriptstyle\mathtt{\sim}$}}

%*****************************************************************************%
%		 			Bold fonts
%*****************************************************************************%
\newcommand{\Ba} {\xmath{\bmath{a}}}
\newcommand{\Bb} {\xmath{\bmath{b}}}
\newcommand{\Bc} {\xmath{\bmath{c}}}
\newcommand{\Bd} {\xmath{\bmath{d}}}
\newcommand{\Be} {\xmath{\bmath{e}}}
\newcommand{\Bf} {\xmath{\bmath{f}}}
\newcommand{\Bg} {\xmath{\bmath{g}}}
\newcommand{\Bh} {\xmath{\bmath{h}}}
\newcommand{\Bi} {\xmath{\bmath{i}}}
\newcommand{\Bj} {\xmath{\bmath{j}}}
\newcommand{\Bk} {\xmath{\bmath{k}}}
\newcommand{\Bl} {\xmath{\bmath{l}}}
\newcommand{\Bm} {\xmath{\bmath{m}}}
\newcommand{\Bn} {\xmath{\bmath{n}}}
\newcommand{\Bo} {\xmath{\bmath{o}}}
\newcommand{\Bp} {\xmath{\bmath{p}}}
\newcommand{\Bq} {\xmath{\bmath{q}}}
\newcommand{\Br} {\xmath{\bmath{r}}}
\newcommand{\Bs} {\xmath{\bmath{s}}}
\newcommand{\Bt} {\xmath{\bmath{t}}}
\newcommand{\Bu} {\xmath{\bmath{u}}}
\newcommand{\Bv} {\xmath{\bmath{v}}}
\newcommand{\Bw} {\xmath{\bmath{w}}}
\newcommand{\Bx} {\xmath{\bmath{x}}}
\newcommand{\By} {\xmath{\bmath{y}}}
\newcommand{\Bz} {\xmath{\bmath{z}}}

\newcommand{\BA} {\xmath{\bmath{A}}}
\newcommand{\BB} {\xmath{\bmath{B}}}
\newcommand{\BC} {\xmath{\bmath{C}}}
\newcommand{\BD} {\xmath{\bmath{D}}}
\newcommand{\BE} {\xmath{\bmath{E}}}
\newcommand{\BF} {\xmath{\bmath{F}}}
\newcommand{\BG} {\xmath{\bmath{G}}}
\newcommand{\BH} {\xmath{\bmath{H}}}
\newcommand{\BI} {\xmath{\bmath{I}}}
\newcommand{\BJ} {\xmath{\bmath{J}}}
\newcommand{\BK} {\xmath{\bmath{K}}}
\newcommand{\BL} {\xmath{\bmath{L}}}
\newcommand{\BM} {\xmath{\bmath{M}}}
\newcommand{\BN} {\xmath{\bmath{N}}}
\newcommand{\BO} {\xmath{\bmath{O}}}
\newcommand{\BP} {\xmath{\bmath{P}}}
\newcommand{\BQ} {\xmath{\bmath{Q}}}
\newcommand{\BR} {\xmath{\bmath{R}}}
\newcommand{\BS} {\xmath{\bmath{S}}}
\newcommand{\BT} {\xmath{\bmath{T}}}
\newcommand{\BU} {\xmath{\bmath{U}}}
\newcommand{\BV} {\xmath{\bmath{V}}}
\newcommand{\BW} {\xmath{\bmath{W}}}
\newcommand{\BX} {\xmath{\bmath{X}}}
\newcommand{\BY} {\xmath{\bmath{Y}}}
\newcommand{\BZ} {\xmath{\bmath{Z}}}

\newcommand{\Balpha}		{\xmath{\bmath{\alpha}}}
\newcommand{\BAlpha}		{\xmath{\bmath{\Alpha}}}
\newcommand{\Bbeta}			{\xmath{\bmath{\beta}}}
\newcommand{\BBeta}			{\xmath{\bmath{\Beta}}}
\newcommand{\Bgamma}		{\xmath{\bmath{\gamma}}}
\newcommand{\BGamma}		{\xmath{\bmath{\Gamma}}}
\newcommand{\Bdelta}		{\xmath{\bmath{\delta}}}
\newcommand{\BDelta}		{\xmath{\bmath{\Delta}}}
\newcommand{\Bepsilon}		{\xmath{\bmath{\epsilon}}}
\newcommand{\BEpsilon}		{\xmath{\bmath{\Epsilon}}}
\newcommand{\Bvarepsilon}		{\xmath{\bmath{\varepsilon}}}
\newcommand{\Bzeta}			{\xmath{\bmath{\zeta}}}
\newcommand{\BZeta}			{\xmath{\bmath{\Zeta}}}
%	\newcommand{\BBeta}			{\xmath{\bmath{\eta}}} % \Beta already defined...
\newcommand{\BEta}			{\xmath{\bmath{\Eta}}}
\newcommand{\Btheta}		{\xmath{\bmath{\theta}}}
\newcommand{\BTheta}		{\xmath{\bmath{\Theta}}}
\newcommand{\Biota}			{\xmath{\bmath{\iota}}}
\newcommand{\BIota}			{\xmath{\bmath{\Iota}}}
\newcommand{\Bkappa}		{\xmath{\bmath{\kappa}}}
\newcommand{\BKappa}		{\xmath{\bmath{\Kappa}}}
\newcommand{\Blambda}		{\xmath{\bmath{\lambda}}}
\newcommand{\BLambda}		{\xmath{\bmath{\Lambda}}}
\newcommand{\Bmu}			{\xmath{\bmath{\mu}}}
\newcommand{\BMu}			{\xmath{\bmath{\Mu}}}
\newcommand{\Bnu}			{\xmath{\bmath{\nu}}}
\newcommand{\BNu}			{\xmath{\bmath{\Nu}}}
\newcommand{\Bxi}			{\xmath{\bmath{\xi}}}
\newcommand{\BXi}			{\xmath{\bmath{\Xi}}}
\newcommand{\Bpi}			{\xmath{\bmath{\pi}}}
\newcommand{\BPi}			{\xmath{\bmath{\Pi}}}
\newcommand{\Brho}			{\xmath{\bmath{\rho}}}
\newcommand{\BRho}			{\xmath{\bmath{\Rho}}}
\newcommand{\Bsigma}		{\xmath{\bmath{\sigma}}}
\newcommand{\BSigma}		{\xmath{\bmath{\Sigma}}}
\newcommand{\Btau}			{\xmath{\bmath{\tau}}}
\newcommand{\BTau}			{\xmath{\bmath{\Tau}}}
\newcommand{\Bupsilon}		{\xmath{\bmath{\upsilon}}}
\newcommand{\BUpsilon}		{\xmath{\bmath{\Upsilon}}}
\newcommand{\Bphi}			{\xmath{\bmath{\phi}}}
\newcommand{\BPhi}			{\xmath{\bmath{\Phi}}}
\newcommand{\Bchi}			{\xmath{\bmath{\chi}}}
\newcommand{\BChi}			{\xmath{\bmath{\Chi}}}
\newcommand{\Bpsi}			{\xmath{\bmath{\psi}}}
\newcommand{\BPsi}			{\xmath{\bmath{\Psi}}}
\newcommand{\Bomega}		{\xmath{\bmath{\omega}}}
\newcommand{\BOmega}		{\xmath{\bmath{\Omega}}}

%*****************************************************************************%
%		 			 my "lazy" commands
%*****************************************************************************%
\newcommand{\tbold}[1]{\textbf{#1}}
\newcommand{\tdef}[1]		{{\color{mygreen}{\textbf{#1}}}}

\newcommand{\w}			{{\bmath{w}}\xspace}
\newcommand{\wmin}			{{\bmath{w^*}}\xspace}
\renewcommand{\b}			{{\bmath{b}}\xspace}
\newcommand{\e}			{{\bmath{e}}\xspace}
\newcommand{\x}			{{\bmath{x}}\xspace}
\newcommand{\xtil}			{{\bmath{\tilde{x}}}\xspace}
\renewcommand{\r}			{{\bmath{r}}\xspace}
\newcommand{\s}			{{\bmath{s}}\xspace}
\newcommand{\q}			{{\bmath{q}}\xspace}
\newcommand{\wtil}			{{\bmath{\widetilde{w}}}\xspace}
\renewcommand{\u}			{{\bmath{u}}\xspace}

% follow the convention that a -> 1, b -> 2, c -> 3...
\newcommand{\va}			{{\bmath{v_1}}\xspace}
\newcommand{\vb}			{{\bmath{v_2}}\xspace}
\newcommand{\vc}			{{\bmath{v_3}}\xspace}
\newcommand{\vd}			{{\bmath{v_4}}\xspace}
\newcommand{\ua}			{{\bmath{u_1}}\xspace}
\newcommand{\ub}			{{\bmath{u_2}}\xspace}
\newcommand{\uc}			{{\bmath{u_3}}\xspace}
\newcommand{\ud}			{{\bmath{u_4}}\xspace}

\newcommand{\AL}			{{\bmath{L_\rho}}\xspace}
\newcommand{\Ctil}			{{\bmath{\widetilde{C}}}\xspace}

\newcommand{\A}			{{\bmath{A}}\xspace}
\newcommand{\B}			{{\bmath{B}}\xspace}
\newcommand{\C}			{{\bmath{C}}\xspace}
\newcommand{\F}			{{\bmath{F}}\xspace}
\newcommand{\Q}			{{\bmath{Q}}\xspace}
\newcommand{\I}			{{\bmath{I}}\xspace}
\newcommand{\U}			{{\bmath{U}}\xspace}
\newcommand{\X}			{{\bmath{X}}\xspace}
\newcommand{\Y}			{{\bmath{Y}}\xspace}
\newcommand{\YXw}			{{\bmath{YXw}}\xspace}
\newcommand{\Xw}			{{\bmath{Xw}}\xspace}
\newcommand{\bzero}			{{\bmath{0}}\xspace}

% bar'ed notations for the canonical admm definition
\newcommand{\Abar}			{{\bmath{\bar{A}}}\xspace}
\newcommand{\Bbar}			{{\bmath{\bar{B}}}\xspace}
\newcommand{\xbar}			{{\bmath{\bar{x}}}\xspace}
\newcommand{\ybar}			{{\bmath{\bar{y}}}\xspace}
\newcommand{\bbar}			{{\bmath{\bar{b}}}\xspace}
\newcommand{\fbar}			{{\bmath{\bar{f}}}\xspace}
\newcommand{\gbar}			{{\bmath{\bar{g}}}\xspace}
\newcommand{\ubar}			{{\bmath{\bar{u}}}\xspace}
\newcommand{\Fbar}			{{\bmath{\bar{F}}}\xspace}
\newcommand{\Gbar}			{{\bmath{\bar{G}}}\xspace}

\newcommand{\prox}		{\xmath{\mathrm{Prox}}} % meh, good enough
\newcommand{\soft}		{\xmath{\mathrm{Soft}}}

% loss functions
\newcommand{\loss}		{\xmath{\ell}}
\newcommand{\Loss}		{\xmath{\mathcal{L}}}

% regularizer
\newcommand{\Reg}{\xmath{\mathcal{R}}}

\newcommand{\kron}{\xmath{\bigotimes}}

% ell related stufs
\newcommand{\elltwo} {\xmath{\ell_2}}
\newcommand{\ellone} {\xmath{\ell_1}}
\newcommand{\ellzero}{\xmath{\ell_0}}
\newcommand{\ellinf} {\xmath{\ell_\infty}}

\newcommand{\normsq}[1]{\norm{#1}^2}

\newcommand{\sign}[1] {\text{sign}\left(#1\right)} 

\newcommand{\iter}{^{(t)}}
\newcommand{\iterp}{^{(t+1)}}

% for canoncial admm primal variable size
\newcommand{\pbar}{{\xmath{\bar{p}}}}
\newcommand{\qbar}{{\xmath{\bar{q}}}}


\newcommand{\ptil}{{\xmath{\tilde{p}}}}
\newcommand{\dtil}{{\xmath{\tilde{d}}}}

\newcommand{\xstar} {\xmath{\x^*}}
\newcommand{\pstar} {{\xmath{p^*}}}

% ADMM termination criteria 
\newcommand{\varepsADMM}{\varepsilon}

%************* line number equation fix *************************
\usepackage{lineno}
% http://tex.stackexchange.com/questions/43648/why-doesnt-lineno-number-a-paragraph-when-it-is-followed-by-an-align-equation?rq=1
% http://phaseportrait.blogspot.com/2007/08/lineno-and-amsmath-compatibility.html
\newcommand*\patchAmsMathEnvironmentForLineno[1]{%
  \expandafter\let\csname old#1\expandafter\endcsname\csname #1\endcsname
  \expandafter\let\csname oldend#1\expandafter\endcsname\csname end#1\endcsname
  \renewenvironment{#1}%
     {\linenomath\csname old#1\endcsname}%
     {\csname oldend#1\endcsname\endlinenomath}}% 
\newcommand*\patchBothAmsMathEnvironmentsForLineno[1]{%
  \patchAmsMathEnvironmentForLineno{#1}%
  \patchAmsMathEnvironmentForLineno{#1*}}%
\AtBeginDocument{%
\patchBothAmsMathEnvironmentsForLineno{equation}%
\patchBothAmsMathEnvironmentsForLineno{align}%
\patchBothAmsMathEnvironmentsForLineno{flalign}%
\patchBothAmsMathEnvironmentsForLineno{alignat}%
\patchBothAmsMathEnvironmentsForLineno{gather}%
\patchBothAmsMathEnvironmentsForLineno{multline}%
}
%************************************************************

