%!TEX root = paper.tex
\begin{frontmatter}
\title{Disease Prediction based on Functional Connectomes using a \\ Scalable and Spatially-Informed Support Vector Machine}
\author[eecs]{Takanori Watanabe\corref{corresponding}}\ead{takanori@umich.edu}
\author[psych]{Daniel Kessler}\ead{kesslerd@umich.edu}
\author[eecs,stats]{Clayton Scott}\ead{clayscot@umich.edu}
\author[psych]{Michael Angstadt}\ead{mangstad@med.umich.edu}
\author[psych]{Chandra Sripada}\ead{sripada@umich.edu}

\address[eecs]{Department of Electrical Engineering and Computer Science, University of Michigan, Ann Arbor, MI, USA}

\address[stats]{Department of Statistics, University of Michigan, Ann Arbor, MI, USA}

\address[psych]{Department of Psychiatry, University of Michigan, Ann Arbor, MI, USA}
%===================================================================%
% Corresponding author. 
%===================================================================%
\cortext[corresponding]{
	Corresponding author.
	1301 Beal Avenue, 4111 EECS, Ann Arbor, MI, 48109 USA;
	Tel: +1 734 615 7027
}

%===================================================================%
% Abstract 
%===================================================================%

\begin{abstract}
Substantial evidence indicates that major psychiatric disorders are associated with distributed neural dysconnectivity, leading to strong interest in using neuroimaging methods to accurately predict disorder status.
In this work, we are specifically interested in a multivariate approach that uses features derived from whole-brain resting state functional connectomes.
However, functional connectomes reside in a high dimensional space, which complicates model interpretation and introduces numerous statistical and computational challenges. 
Traditional feature selection techniques are used to reduce data dimensionality, but are blind to the spatial structure of the connectomes.
We propose a regularization framework where the $6$-D structure of the functional connectome (defined by pairs of points in $3$-D space) is explicitly taken into account via the fused Lasso or the \mbox{GraphNet} regularizer.
Our method only restricts the loss function to be convex and margin-based, allowing non-differentiable loss functions such as the hinge-loss to be used.
Using the fused Lasso or GraphNet regularizer with the hinge-loss leads to a structured sparse support vector machine (SVM) with embedded feature selection.
We introduce a novel efficient optimization algorithm based on augmented Lagrangian and the classical alternating direction method, which can solve both fused Lasso and GraphNet regularized SVM with very little modification.
We also demonstrate that the inner subproblems of the algorithm can be solved efficiently in analytic form by coupling the variable splitting strategy with a data augmentation scheme.
Experiments on simulated data and resting state scans from a large schizophrenia dataset show that our proposed approach can identify predictive regions that are spatially contiguous in the $6$-D ``connectome space,'' offering an additional layer of interpretability that could provide new insights about various disease processes.
\end{abstract}

%=============================================================================%
% Keywords (max 6 words)
%=============================================================================%
\begin{keyword}
Classification \sep
feature selection \sep
structured sparsity \sep
resting state fMRI \sep
functional connectivity \sep
support vector machine
\end{keyword}

\end{frontmatter}
